% equivalence-relation.tex

%%%%%%%%%%%%%%%
\begin{frame}{}
  \begin{center}
    \teal{\Large Equivalence Relations}
  \end{center}
\end{frame}
%%%%%%%%%%%%%%%

%%%%%%%%%%%%%%%
\begin{frame}{}
  \begin{definition}[Equivalence Relation]
    $R$ is an {\it equivalence relation} on $X$ iff $R$ is 
    \begin{itemize}
      \item reflexive 
      \item symmetric
      \item transitive
    \end{itemize}
  \end{definition}

  \pause
  \[
    =\; \in \mathbb{R} \times \mathbb{R}
  \]

  \pause
  \[
    \parallel\; \in \mathbb{L} \times \mathbb{L}
  \]

  \pause
  \[
    a \sim b \iff a \;\%\; 12 = b \;\%\; 12
  \]

  \pause
  \begin{center}
    \red{\large Why are equivalence relations important?}
  \end{center}
\end{frame}
%%%%%%%%%%%%%%%

%%%%%%%%%%%%%%%
\begin{frame}{}
  \begin{center}
    \blue{\large Equivalence Relations as \red{Abstractions}}
  \end{center}

  \begin{columns}
    \column{0.50\textwidth}
      \pause
      \fig{width = 0.50\textwidth}{figs/modulo-clock}
    \column{0.50\textwidth}
      \pause
      \fig{width = 0.80\textwidth}{figs/conference-hall-group}
      \begin{center}
	``全国人民代表大会各省代表团''
      \end{center}
  \end{columns}

  \pause
  \vspace{0.60cm}
  \[
    \red{\text{Equivalence Relation} \iff \text{Partition}}
  \]
\end{frame}
%%%%%%%%%%%%%%%

%%%%%%%%%%%%%%%
\begin{frame}{}
  \centerline{\LARGE Partition}

  \vspace{0.60cm}
  \fig{width = 0.60\textwidth}{figs/partition}

  \begin{center}
    ``不空、不漏、不重''
  \end{center}
\end{frame}
%%%%%%%%%%%%%%%

%%%%%%%%%%%%%%%
\begin{frame}{}
  \begin{definition}[Partition]
    A family of sets \red{$\set{A_{\alpha}: \alpha \in I}$} is a \blue{\it partition} of $X$ if

    \begin{enumerate}[(i)]
      \item 
	\[
	  \forall \alpha \in I: A_{\alpha} \neq \emptyset
	\]
	\[
	  \uncover<2->{\teal{(\forall \alpha \in I \; \exists x \in X: x \in A_{\alpha})}}
	\]
      \item 
	\[
	  \bigcup_{\alpha \in I} A_{\alpha} = X
	\]
	\[
	  \uncover<3->{\teal{(\forall x \in X \; \exists \alpha \in I: x \in A_{\alpha})}}
	\]
      \item 
	\[
	  \forall \alpha, \beta \in I: A_{\alpha} \cap A_{\beta} = \emptyset \lor A_{\alpha} = A_{\beta}
	\]
	\[
	  \uncover<4->{\teal{(\forall \alpha, \beta \in I: A_{\alpha} \cap A_{\beta} \neq \emptyset \implies A_{\alpha} = A_{\beta})}}
	\]
    \end{enumerate}
  \end{definition}
\end{frame}
%%%%%%%%%%%%%%%

%%%%%%%%%%%%%%%
\begin{frame}{}
  \[
    \red{\text{Equivalence Relation } R \subseteq X \times X \implies \text{Partition } \Pi \text{ of } X}
  \]

  \pause
  \begin{definition}[Equivalence Class]
    The {\it equivalence class} of $a$ {\it modulo} $R$ is a \purple{set}:
    \[
      \red{[a]_{R}} = \set{b \in X: a R b}
    \]
  \end{definition}

  \pause
  \vspace{0.60cm}
  \begin{definition}[Quotient Set]
    The {\it quotient set} is a \purple{set}:
    \[
      X/R = \set{[a]_{R} \mid a \in X}
    \]
  \end{definition}
\end{frame}
%%%%%%%%%%%%%%%

%%%%%%%%%%%%%%%
\begin{frame}{}
  \begin{theorem}
    \[
      X/R = \set{[a]_{R} \mid a \in X} \text{ is a partition of } X.
    \]
  \end{theorem}

  \pause
  \[
    \forall a \in X: [a]_{R} \neq \emptyset
  \]

  \pause
  \[
    \forall a \in X: \exists b \in X: a \in [b]_{R}
  \]

  \pause
  \begin{theorem}
    \[
      \forall a \in X, b \in X: [a]_{R} \cap [b]_{R} = \emptyset \lor [a]_{R} = [b]_{R}
    \]
  \end{theorem}

  \pause
  \[
    \blue{\forall a \in X, b \in X: [a]_{R} \cap [b]_{R} \neq \emptyset \implies [a]_{R} = [b]_{R}}
  \]
\end{frame}
%%%%%%%%%%%%%%%

%%%%%%%%%%%%%%%
% \begin{frame}{}
%   \begin{exampleblock}{Equivalence Relation (UD Problem $10.10$)}
%     \[
%       \forall a, b \in X: [a]_{R} = [b]_{R} \iff a R b.
%     \]
%   \end{exampleblock}
% \end{frame}
%%%%%%%%%%%%%%%

%%%%%%%%%%%%%%%
\begin{frame}{}
  \[
    \red{\text{Partition } \Pi \text{ of } X \implies \text{Equivalence Relation } R \subseteq X \times X}
  \]

  \pause
  \begin{definition}
    \[
      (a, b) \in R \iff \exists S \in \Pi: a \in S \land b \in S
    \]

    \pause
    \vspace{-0.20cm}
    \[
      R = \set{(a, b) \in X \times X \mid \exists S \in \Pi: a \in S \land b \in S}
    \]
  \end{definition}

  \pause
  \begin{theorem}
    \[
      R \text{ is an equivalence relation on } X.
    \]
  \end{theorem}

  \pause
  \[
    \forall x \in X: xRx
  \]

  \pause
  \[
    \forall x, y \in X: xRy \implies yRx
  \]

  \pause
  \[
    \forall x, y, z \in X: xRy \land yRz \implies xRz
  \]
\end{frame}
%%%%%%%%%%%%%%%

%%%%%%%%%%%%%%%
\begin{frame}{}
  \fig{width = 0.60\textwidth}{figs/partition}

  \[
    \red{\text{Equivalence Relation} \iff \text{Partition}}
  \]
\end{frame}
%%%%%%%%%%%%%%%

%%%%%%%%%%%%%%%
\begin{frame}{}
  \begin{definition}
    \[
      \sim\; \subseteq \mathbb{N}^2 \times \mathbb{N}^2
    \]

    \[
      (a, b) \sim (c, d) \iff a +_{\mathbb{N}} d = b +_{\mathbb{N}} c
    \]
  \end{definition}

  \pause
  \vspace{0.30cm}
  \begin{theorem}
    $\sim$ is an equivalence relation.
  \end{theorem}

  \pause
  \vspace{0.30cm}
  \begin{center}
    \red{$Q:$ What is $\mathbb{N} \times \mathbb{N}/\sim$?}
  \end{center}

  \pause
  \begin{definition}[$\mathbb{Z}$]
    \[
      \mathbb{Z} \triangleq \mathbb{N} \times \mathbb{N}/\sim
    \]
  \end{definition}

  \pause
  \[
    [(1, 3)]_{\sim} = \set{(0, 2), (1, 3), (2, 4), (3, 5), \cdots} \triangleq -2 \in \mathbb{Z}
  \]
\end{frame}
%%%%%%%%%%%%%%%

%%%%%%%%%%%%%%%
\begin{frame}{}
  \fig{width = 0.50\textwidth}{figs/integer-definition}
  \[
    \mathbb{Z} \triangleq \mathbb{N} \times \mathbb{N}/\sim
  \]
\end{frame}
%%%%%%%%%%%%%%%

%%%%%%%%%%%%%%%
\begin{frame}{}
  \begin{definition}[$+_\mathbb{Z}$]
    \[
      [(m_1, n_1)] +_{\mathbb{Z}} [(m_2, n_2)] = [m_1 +_{\mathbb{N}} m2, n_1 +_{\mathbb{N}} n_2]
    \]
  \end{definition}

  \pause
  \vspace{0.60cm}
  \begin{definition}[$\cdot_\mathbb{Z}$]
    \begin{gather*}
      [(m_1, n_1)] \cdot_{\mathbb{Z}} [(m_2, n_2)] \\
      = [m_1 \cdot_{\mathbb{N}} m_2 +_{\mathbb{N}} n_1 \cdot_{\mathbb{N}} n_2, 
         m_1 \cdot_{\mathbb{N}} n_2 +_{\mathbb{N}} n_1 \cdot_{\mathbb{N}} m_2]
    \end{gather*}
  \end{definition}
\end{frame}
%%%%%%%%%%%%%%%

%%%%%%%%%%%%%%%
\begin{frame}{}
  \begin{definition}
    \[
      \sim\; \subseteq (\mathbb{Z} \times \mathbb{Z} \setminus \set{0_{\mathbb{Z}}})^{2}
    \]

    \[
      (a, b) \sim (c, d) \iff a \cdot_{\mathbb{Z}} d = b \cdot_{\mathbb{Z}} c
    \]
  \end{definition}

  \pause
  \vspace{0.30cm}
  \begin{definition}[$\mathbb{Q}$]
    \[
      \mathbb{Q} \triangleq \mathbb{Z} \times \mathbb{Z}/\sim
    \]
  \end{definition}
\end{frame}
%%%%%%%%%%%%%%%

%%%%%%%%%%%%%%%
\begin{frame}{}
  \fig{width = 0.50\textwidth}{figs/rational-definition}
  \[
    \mathbb{Q} \triangleq \mathbb{Z} \times \mathbb{Z}/\sim
  \]
\end{frame}
%%%%%%%%%%%%%%%

%%%%%%%%%%%%%%%
\begin{frame}{}
  \begin{center}
    \red{How to define $\mathbb{R}$ as equivalence classes of ordered pairs of $\mathbb{Q}$?}
  \end{center}

  \pause
  \fig{width = 0.40\textwidth}{figs/stay-tuned}
\end{frame}
%%%%%%%%%%%%%%%
