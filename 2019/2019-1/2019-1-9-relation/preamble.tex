% preamble.tex

\usepackage{lmodern}

\usepackage{xeCJK}
\usetheme{CambridgeUS} % try Madrid, Pittsburgh
\usecolortheme{beaver}
\usefonttheme[]{serif} % try "professionalfonts"

\setbeamertemplate{itemize items}[default]
\setbeamertemplate{enumerate items}[default]

\usepackage{amsmath, amsfonts, latexsym, mathtools}
\newcommand{\set}[1]{\{#1\}}
\newcommand{\bset}[1]{\big\{#1\big\}}
\newcommand{\Bset}[1]{\Big\{#1\Big\}}
\DeclareMathOperator*{\argmin}{\arg\!\min}

\definecolor{bgcolor}{rgb}{0.95,0.95,0.92}

\usepackage{listings}
\lstdefinestyle{CStyle}{
    language = C,
    basicstyle = \ttfamily\bfseries,
    backgroundcolor = \color{bgcolor},   
    keywordstyle = \color{blue},
    stringstyle = \color{red},
    commentstyle = \color{cyan},
    breakatwhitespace = false,
    breaklines = true,                 
    mathescape = true,
    escapeinside = ||,
    morekeywords = {repeat, until},
    showspaces = false,                
    showstringspaces = false,
    showtabs = false,                  
}

% colors
\newcommand{\red}[1]{\textcolor{red}{#1}}
\newcommand{\redoverlay}[2]{\textcolor<#2>{red}{#1}}
\newcommand{\green}[1]{\textcolor{green}{#1}}
\newcommand{\blue}[1]{\textcolor{blue}{#1}}
\newcommand{\blueoverlay}[2]{\textcolor<#2>{blue}{#1}}
\newcommand{\teal}[1]{\textcolor{teal}{#1}}
\newcommand{\purple}[1]{\textcolor{purple}{#1}}
\newcommand{\cyan}[1]{\textcolor{cyan}{#1}}

% colorded box
\newcommand{\rbox}[1]{\red{\boxed{#1}}}
\newcommand{\gbox}[1]{\green{\boxed{#1}}}
\newcommand{\bbox}[1]{\blue{\boxed{#1}}}
\newcommand{\pbox}[1]{\purple{\boxed{#1}}}

\usepackage{pifont}
\usepackage{wasysym}

\newcommand{\cmark}{\green{\ding{51}}}
\newcommand{\xmark}{\red{\ding{55}}}
%%%%%%%%%%%%%%%%%%%%%%%%%%%%%%%%%%%%%%%%%%%%%%%%%%%%%%%%%%%%%%
% for fig without caption: #1: width/size; #2: fig file
\newcommand{\fig}[2]{
  \begin{figure}[htp]
    \centering
      \includegraphics[#1]{#2}
  \end{figure}
}

\newcommand{\titletext}{1-9 Set Theory (II): Relations}

\newcommand{\lset}{\mathcal{L}_{\textsl{Set}}}
\newcommand{\ps}[1]{\mathcal{P}(#1)}
\newtheorem{axiom}{Axiom}

\newcommand{\thankyou}{
\begin{frame}[noframenumbering]{}
  \fig{width = 0.50\textwidth}{figs/thankyou.png}
\end{frame}
}