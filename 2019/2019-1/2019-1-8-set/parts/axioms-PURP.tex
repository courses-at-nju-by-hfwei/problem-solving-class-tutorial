% axioms-PURP.tex

%%%%%%%%%%%%%%%
\begin{frame}{}
  \begin{axiom}[Paring Axiom]
    For any sets $x$ and $y$, there is a set having as members just $x$ and $y$:

    \[
      \forall x\; \forall y\; \red{\exists B}\; \big(\forall z (z \in B \iff z = x \lor z = y)\big).
    \]
  \end{axiom}

  \pause
  \begin{definition}[``$\set{x, y}$'']
    \[
      \set{x, y} \triangleq \text{ the \teal{unique} set obtained by \blue{paring} } x \text{ and } y.
    \]
  \end{definition}

  \pause
  \begin{theorem}
    \[
      \set{x, y} = \set{y, x}.
    \]
  \end{theorem}

  \pause
  \begin{definition}[``$\set{x}$'']
    \[
      \set{x} \triangleq \set{x, x}.
    \]
  \end{definition}
\end{frame}
%%%%%%%%%%%%%%%

%%%%%%%%%%%%%%%
\begin{frame}{}
  \begin{axiom}[Union Axiom (Simplified Version)]
    For any sets $x$ and $y$, there is a set whose members are
    the elements belonging either to $x$ or to $y$ (or both):

    \[
      \forall x\; \forall y\; \red{\exists B}\; \big(\forall z (z \in B \iff z \;\blue{\in}\; x \lor z \;\blue{\in}\; y)\big).
    \]
  \end{axiom}

  \pause
  \vspace{0.60cm}
  \begin{definition}[``$x \cup y$'']
    \[
      x \cup y \triangleq \text{ the \teal{unique} set obtained by \blue{unioning} } x \text{ and } y.
    \]
  \end{definition}
\end{frame}
%%%%%%%%%%%%%%%

%%%%%%%%%%%%%%%
\begin{frame}{}
  \begin{definition}[``$\set{x, y}$'']
    \[
      \set{x, y} \triangleq \text{ the \teal{unique} set obtained by \blue{paring} } x \text{ and } y.
    \]
  \end{definition}

  \begin{definition}[``$\set{x}$'']
    \[
      \set{x} \triangleq \set{x, x}.
    \]
  \end{definition}

  \pause
  \vspace{0.60cm}
  \begin{definition}[``$\set{x, y, z}$'']
    \[
      \set{x, y, z} \triangleq \set{x, y} \cup \set{z}.
    \]
  \end{definition}

  \pause
  \begin{center}
    We can use \blue{pairing} and \blue{union} together to form \red{finite sets}.
  \end{center}
\end{frame}
%%%%%%%%%%%%%%%

%%%%%%%%%%%%%%%
\begin{frame}{}
  \begin{axiom}[Union Axiom (Extended Version)]
    For any set $A$, there is a set $B$ such that:
    \[
      \forall x\; (x \in B \iff x \text{ belongs to some member of } A).
    \]
    \[
      \forall x \big( x \in B \iff \exists y \in A (x \in y) \big).
    \]
  \end{axiom}

  \pause
  \begin{definition}[``$\bigcup A$'' (Arbitrary Union)]
    \[
      \bigcup A \triangleq \text{ the \teal{unique} set obtained by \blue{unioning} } A.
    \]
  \end{definition}

  \pause
  \begin{theorem}
    \[
      \bigcup \set{x, y} = x \cup y.
    \]
  \end{theorem}

  \pause
  \begin{theorem}
    \[
      \bigcup \emptyset = \emptyset.
    \]
  \end{theorem}
\end{frame}
%%%%%%%%%%%%%%%

%%%%%%%%%%%%%%%
\begin{frame}{}
  \begin{axiom}[Replacement Axioms (Simplified Version: Subset Axioms; Separation Axioms)]
    \red{Let $\psi$ be a predicate.}
    For any set $u$, there is a set $B$
    which is a subset of $u$ such that each element $x$ of $B$ satisfies $\psi(x)$:

    \[
      \forall u\; \red{\exists B}\; \big( \forall x (x \in B \iff x \in u \land \psi(x)) \big).
    \]
  \end{axiom}

  \begin{definition}[``$\set{x \in u \mid \psi(x)}$'']
    \[
      \set{x \in u \mid \psi(x)} \triangleq \text{ the \teal{unique} set obtained by \blue{separating} from } u \text{ with } \psi.
    \]
  \end{definition}

  \pause
  \begin{definition}[``$u \cap v$'']
    \[
      u \cap v \triangleq \set{x \in u \mid x \in v}.
    \]
  \end{definition}
\end{frame}
%%%%%%%%%%%%%%%

%%%%%%%%%%%%%%%
\begin{frame}{}
  \begin{theorem}[``$\bigcap A$'' (Arbitrary Intersection)]
    \red{For any nonempty set $A$}, there is a unique set $B$ such that
    \[
      \forall x\; (x \in B \iff x \text{ belongs to every member of } A).
    \]
    \[
      \forall x\; \big(x \in B \iff \forall y \in A (x \in y) \big).
    \]
  \end{theorem}

  \pause
  \begin{proof}
    \begin{center}
      Let $c$ be a fixed member of $A$.
    \end{center}

    \pause
    \vspace{-0.50cm}
    \[
      \bigcap A \triangleq \set{x \in c \mid x \text{ belongs to every other member of } A}.
    \]
  \end{proof}

  \pause
  \begin{alertblock}{``$\bigcap \emptyset$''}
    \[
      \bigcap \emptyset \text{ is \red{\it not} a set}.
    \]
  \end{alertblock}
\end{frame}
%%%%%%%%%%%%%%%

%%%%%%%%%%%%%%%
\begin{frame}{}
  \begin{theorem}[No Universal Set]
    There is no universal set.

    \[
      \red{\nexists B} \big(\forall x (x \in B) \big).
    \]
  \end{theorem}

  \pause
  \begin{proof}
    \begin{center}
      \red{For any set $A$, we construct a set not in $A$.}
    \end{center}

    \pause
    \[
      \blue{B = \set{x \in A \mid x \notin x}}
    \]
    \pause
    \vspace{-0.30cm}
    \[
      B \in B \iff B \in A \land B \notin B
    \]

    \pause
    \[
      \red{\boxed{B \notin A}}
    \]

    \pause
    \vspace{-0.30cm}
    \[
      B \in A \implies (B \in B \iff B \notin B)
    \]
  \end{proof}
\end{frame}
%%%%%%%%%%%%%%%

%%%%%%%%%%%%%%%
% \begin{frame}{}
%   \begin{theorem}[Russell's Paradox]
%     \[
%       \set{x \mid x \notin x} \text{ is \red{\it not} a set.}
%     \]
%   \end{theorem}
% 
%   \pause
%   \[
%     B = \set{x \in A \mid x \notin x}
%   \]
%   \begin{center}
%     does \red{\it not} lead to contradiction.
%   \end{center}
% \end{frame}
%%%%%%%%%%%%%%%

%%%%%%%%%%%%%%%
\begin{frame}{}
  \begin{definition}[``$u \setminus v$'' \red{(Relative Complement)}]
    \[
      u \setminus v \triangleq \set{x \in u \mid x \notin v}.
    \]
  \end{definition}

  \pause
  \begin{theorem}[No ``Absolute Complement'']
    For any set $B$, the following is \red{\it not} a set:
    \[
      \set{x \mid x \notin B}.
    \]
  \end{theorem}

  \pause
  \begin{proof}
    \begin{center}
      \red{By Contradiction.}
    \end{center}

    \pause
    \vspace{-0.50cm}
    \[
      \set{x \mid x \notin B} \cup B \text{ would be a universal set}.
    \]
  \end{proof}

  \pause
  \begin{quote}
    We can never look for objects ``not in $B$'' \red{unless we know where to start looking}.
    \hfill --- UD (Chapter 6; Page 64)
  \end{quote}
\end{frame}
%%%%%%%%%%%%%%%

%%%%%%%%%%%%%%%
\begin{frame}{}
  \begin{axiom}[Power Set Axiom]
    For any set $A$, there is a set whose members are the subsets of $A$:
    \[
      \forall A\; \red{\exists B}\; \forall x (x \in B \iff x \subseteq A).
    \]
  \end{axiom}

  \pause
  \vspace{0.50cm}
  \begin{definition}[``$\mathcal{P}(A)$'']
    \[
      \mathcal{P}(A) \triangleq \text{ the \teal{unique} power set of } A.
    \]
  \end{definition}

  \pause
  \vspace{0.50cm}
  \begin{alertblock}{The is \red{\it not} correct!}
    \[
      \mathcal{P}(A) \triangleq \set{x \mid x \subseteq A}
    \]
  \end{alertblock}
\end{frame}
%%%%%%%%%%%%%%%
