% cantor-theorem.tex

%%%%%%%%%%%%%%%
\begin{frame}{}
  \begin{center}
    \teal{\Huge Beyond}
  \end{center}

  \[
    \scalebox{8.5}{$\mathfrak{c}$}
  \]
\end{frame}
%%%%%%%%%%%%%%%

%%%%%%%%%%%%%%%
\begin{frame}{}
  \begin{theorem}[Cantor's Theorem (1891)]
    \[
      |A| \neq |\ps{A}|
    \]
  \end{theorem}

  \pause
  \begin{theorem}[Cantor Theorem (ES Theorem 24.4)]
    If $f: A \to \ps{A}$, then $f$ is not onto.
  \end{theorem}

  \pause
  \vspace{0.60cm}
  \fig{width = 0.70\textwidth}{figs/cantor-theorem-proof}
\end{frame}
%%%%%%%%%%%%%%%

%%%%%%%%%%%%%%%
\begin{frame}{}
  \begin{theorem}[Cantor Theorem]
    If $f: A \to \ps{A}$, then $f$ is not onto.
  \end{theorem}

  \vspace{0.60cm}
  \begin{columns}
    \column{0.28\textwidth}
      \fig{width = 1.00\textwidth}{figs/talking-about}
    \pause
    \column{0.25\textwidth}
      \fig{width = 0.90\textwidth}{figs/interesting}
    \pause
    \column{0.25\textwidth}
      \fig{width = 0.80\textwidth}{figs/genius}
    \pause
    \column{0.25\textwidth}
      \fig{width = 1.00\textwidth}{figs/stupid}
  \end{columns}
\end{frame}
%%%%%%%%%%%%%%%

%%%%%%%%%%%%%%%
\begin{frame}{}
  \begin{theorem}[Cantor Theorem]
    If $f: A \to \ps{A}$, then $f$ is not onto.
  \end{theorem}

  \vspace{0.30cm}
  \purple{Understanding this problem:}
  \[
    A = \set{1,2,3}
  \]
  \pause
  \[
    \ps{A} = \Big\{\emptyset, \set{1}, \set{2}, \set{3}, \set{1,2}, \set{1,3}, \set{2,3}, \set{1,2,3}\Big\}
  \]

  \pause
  \begin{description}
    \item[Onto]
      \[
	\forall B \in 2^{A}: \Big(\exists a \in A: f(a) = B\Big)
      \]
    \pause
    \item[Not Onto]
      \[
	\red{\exists} B \in 2^{A}: \Big(\red{\forall} a \in A: f(a) \neq B\Big)
      \]
  \end{description}
\end{frame}
%%%%%%%%%%%%%%%

%%%%%%%%%%%%%%%
\begin{frame}{}
  \begin{theorem}[Cantor Theorem]
    If $f: A \to \ps{A}$, then $f$ is not onto.
  \end{theorem}

  \[
    \red{\exists} B \in \ps{A}: \Big(\red{\forall} a \in A: f(a) \neq B\Big)
  \]

  \pause
  \begin{columns}[t]
    \column{0.50\textwidth}
      \begin{itemize}
	\item<2-> Constructive proof (\red{$\exists$}):
	  \[
	    B = \set{a \in A \mid a \notin f(a)}
	  \]
	\item<4-> By contradiction (\red{$\forall$}):
	  \[
	    \exists a \in A: f(a) = B.
	  \]
      \end{itemize}
    \column{0.40\textwidth}
      \uncover<3->{\fig{width = 0.80\textwidth}{figs/what-is-this}}
  \end{columns}

  \vspace{-0.30cm}
  \uncover<5->{
    \[
      \red{Q: a \in B\emph{?}}
    \]
  }

  \vspace{-0.60cm}
  \uncover<6->{
    \[
      \teal{a \in B \iff a \notin B}
    \]
  }
\end{frame}
%%%%%%%%%%%%%%%

%%%%%%%%%%%%%%%
\begin{frame}{}
  \begin{theorem}[Cantor Theorem]
    If $f: A \to \ps{A}$, then $f$ is not onto.
  \end{theorem}

  \begin{proof}[Diagonal Argument \only<5->{\footnotesize \purple{(以下仅适用于可数集合 $A$)}}]
    \pause
    \begin{table}[]
      \centering
      $\begin{tabu}{|c||c|c|c|c|c|c|}
	\hline
	a      & \multicolumn{6}{c|}{f(a)} \\ \hline
	       & 1      & 2      & 3      & 4      & 5      & \cdots \\ \hline \hline
	1      & \redoverlay{1}{3-}      & 1      & 0      & 0      & 1      & \cdots \\ \hline
	2      & 0      & \redoverlay{0}{3-}      & 0      & 0      & 0      & \cdots \\ \hline
	3      & 1      & 0      & \redoverlay{0}{3-}      & 1      & 0      & \cdots \\ \hline
	4      & 1      & 1      & 1      & \redoverlay{1}{3-}      & 1      & \cdots \\ \hline
	5      & 0      & 1      & 0      & 1      & \redoverlay{0}{3-}      & \cdots \\ \hline
	\vdots & \vdots & \vdots & \vdots & \vdots & \vdots & \cdots \\ \hline
      \end{tabu}$
    \end{table}

    \uncover<4->{
      \[
	B = \blue{\set{0, 1, 1, 0, 1}}
      \]
    }
  \end{proof}
\end{frame}
%%%%%%%%%%%%%%%

%%%%%%%%%%%%%%%
\begin{frame}{}
  \begin{theorem}[Cantor Theorem]
    \[
      |A| \;\red{<}\; |\ps{A}|
    \]
  \end{theorem}

  \pause
  \[
    A \qquad \ps{A} \qquad \ps{\ps{A}} \qquad \ldots
  \]

  \pause
  \vspace{0.60cm}
  \begin{center}
    \red{\large There is no largest infinity.}
  \end{center}
\end{frame}
%%%%%%%%%%%%%%%
