% compare-equal.tex

%%%%%%%%%%%%%%%
\begin{frame}{}
  \fig{width = 0.35\textwidth}{figs/equal-logo}
\end{frame}
%%%%%%%%%%%%%%%

%%%%%%%%%%%%%%%
\begin{frame}{}
  \begin{definition}[$|A| = |B| \;(A \approx B)$ (1878)]
    $A$ and $B$ are \red{\it equipotent} if there exists a \purple{\it bijection} from $A$ to $B$.
  \end{definition}

  \pause
  \[
    \overline{\overline{A}} \quad \text{(two \red{abstractions})}
  \]

  \pause
  \[
    \text{\blue{\small Abstract from elements: }} \set{1,2,3} \quad \text{\it vs.} \quad \set{a,b,c}
  \]

  \pause
  \[
    \text{\blue{\small Abstract from order: }} \set{1,2,3,\cdots} \quad \text{\it vs.} \quad \set{1,3,5,\cdots, 2,4,6,\cdots}
  \]
\end{frame}
%%%%%%%%%%%%%%%

%%%%%%%%%%%%%%%
\begin{frame}{}
  \begin{definition}[$|A| = |B| \;(A \approx B)$ (1878)]
    $A$ and $B$ are \red{\it equipotent} if there exists a \purple{\it bijection} from $A$ to $B$.
  \end{definition}

  \vspace{0.30cm}
  \begin{center}
    \red{$Q:$ Is ``$\approx$'' an equivalence relation?}
  \end{center}

  \pause
  \begin{theorem}[\uncover<3->{\purple{The ``Equivalence Concept'' of Equipotent}}]
    For any sets $A$, $B$, $C$:
    \begin{enumerate}[(a)]
      \item $A \approx B$
      \item $A \approx B \implies B \approx A$
      \item $A \approx B \land B \approx C \implies A \approx C$
    \end{enumerate}
  \end{theorem}
\end{frame}
%%%%%%%%%%%%%%%

%%%%%%%%%%%%%%%
\begin{frame}{}
  \begin{definition}[Finite and Infinite]
    For any set $X$,
    \begin{description}[Infinite]
      \item[Finite] 
	\[
	  \exists n \in \N: |X| = n  \qquad (\red{0 \in \N})
	\]
      \item[Infinite] ($\lnot$ finite)
	\[
	  \forall n \in \N: |X| \neq n
	\]
    \end{description}
  \end{definition}

  % \pause
  % \vspace{0.30cm}
  % \begin{center}
  %   {\red{\it $Q:$ How to prove that a set is infinite?}} \\[8pt]
  %   \pause
  %   {\blue{By contradiction.}}
  % \end{center}
\end{frame}
%%%%%%%%%%%%%%%

%%%%%%%%%%%%%%%
\begin{frame}{}
  \begin{definition}[Finite and Infinite]
    For any set $X$,
    \begin{description}[Uncountable Infinite]
      \item[Countably Infinite]
	\[
	  |X| = |\N| \blue{\;\triangleq \aleph_{0}}
	\]
      \item[Countable]
	\[
	  \text{(finite $\lor$ countably infinite)}
	\]
      \item[Uncountably Infinite] 
	\[
	  \text{($\lnot$ finite) $\land$ \Big($\lnot$ (countably infinite)\Big)}
	\]
	\[
	  \text{($\lnot$ countable)}
	\]
    \end{description}
  \end{definition}
\end{frame}
%%%%%%%%%%%%%%%

%%%%%%%%%%%%%%%
\begin{frame}{}
  \begin{theorem}[$\Z$ is Countable.]
    \[
      |\Z| = |\N|
    \]
  \end{theorem}
\end{frame}
%%%%%%%%%%%%%%%

%%%%%%%%%%%%%%%
\begin{frame}{}
  \begin{theorem}[$\Q$ is Countable. (Cantor 1873-11)]
    \[
      |\Q| = |\N|
    \]
  \end{theorem}

  \pause
  \begin{exampleblock}{$|\Q| = |\N|$ (UD Problem $23.12$)}
    \[
      q \in \red{\Q^{+}}: a / b \; (a,b \in \N^{+})
    \]
    \begin{columns}
      \column{0.50\textwidth}
	\fig{width = 0.50\textwidth}{figs/cantor-monumento}
      \column{0.50\textwidth}
	\fig{width = 0.60\textwidth}{figs/counting-Q}
    \end{columns}
  \end{exampleblock}
\end{frame}
%%%%%%%%%%%%%%%

%%%%%%%%%%%%%%%
\begin{frame}{}
  \begin{theorem}[$\N \times \N$ is Countable.]
    \[
      |\N \times \N| = |\N|
    \]
  \end{theorem}

  \pause
  \[
    f: \N \times \N \to \N 
  \]

  \pause
  \[
    f(m,n) = n + \frac{(m+n)(m+n+1)}{2}
  \]
\end{frame}
%%%%%%%%%%%%%%%

%%%%%%%%%%%%%%%
\begin{frame}{}
  \begin{theorem}[$\R$ is Uncountable. (Cantor 1873-12)]
    \[
      |\R| \neq |\N|
    \]
  \end{theorem}

  \pause
  \fig{width = 0.30\textwidth}{figs/very-important}

  \pause
  \begin{center}
    Different ``Sizes'' of Infinity \\[10pt] \pause
    Cantor's Diagonal Argument (1890)
  \end{center}

\end{frame}
%%%%%%%%%%%%%%%

%%%%%%%%%%%%%%%
\begin{frame}{}
  \begin{theorem}[$\R$ is Uncountable. (Cantor 1873-12)]
    \[
      |\R| \neq |\N|
    \]
  \end{theorem}

  \pause
  \begin{center}
    \red{By Contradiction.}

    \pause
    \[
      f: \R \xleftrightarrow[onto]{1-1} \N
    \]

    \pause
    \fig{width = 0.13\textwidth}{figs/r-uncountable}

    \pause
    \red{By Diagonal Argument.}
  \end{center}
\end{frame}
%%%%%%%%%%%%%%%

\input{parts/cantor-theorem}

%%%%%%%%%%%%%%%
\begin{frame}{}
  \begin{exampleblock}{Infinite Sequences of $0$'s and $1$'s (UD Problem $23.4$)}
    Is the set of all infinite sequences of $0$'s and $1$'s finite,
    countably infinite, or uncountable?
  \end{exampleblock}

  \pause
  \fig{width = 0.35\textwidth}{figs/diagonal-argument-01}
  \begin{center}
    \red{By Diagonal Argument.}
  \end{center}
\end{frame}
%%%%%%%%%%%%%%%

%%%%%%%%%%%%%%%
\begin{frame}{}
  \begin{exampleblock}{Infinite Sequences of $0$'s and $1$'s (UD Problem $23.4$)}
    Is the set of all infinite sequences of $0$'s and $1$'s finite,
    countably infinite, or uncountable?
  \end{exampleblock}

  \[
    f: \set{\set{0,1}^{\ast}} \to \N
  \]

  \pause
  \[
    \purple{f(x_0 x_1 \cdots) = \sum_{i=0}^{\infty} x_i 2^{i}}
  \]

  \pause
  \fig{width = 0.20\textwidth}{figs/wrong}
\end{frame}
%%%%%%%%%%%%%%%

%%%%%%%%%%%%%%%
\begin{frame}{}
  \begin{theorem}[$|\R|$ (Cantor 1877)]
    \[
      |(0,1)| = \red{|\R| = |\R \times \R|} \blue{\;= |\R^{n \in \N}|}
    \]
  \end{theorem}

  \pause
  \begin{proof}
    \pause
    \[
      f(x) = \tan \frac{(2x-1)\pi}{2}
    \]
    \[
      |(0,1)| = |(-\frac{\pi}{2}, \frac{\pi}{2})| = |\R|
    \]

    \pause
    \[
      (x = 0.\blue{a_1 a_2 a_3 \cdots}, y = 0.\red{b_1 b_2 b_3 \cdots}) \pause \mapsto 0.\blue{a_1}\red{b_1}\blue{a_2}\red{b_2}\blue{a_3}\red{b_3}\cdots
    \]
  \end{proof}
\end{frame}
%%%%%%%%%%%%%%%

%%%%%%%%%%%%%%%
\begin{frame}{}
  \begin{theorem}[$|\R|$ (Cantor 1877)]
    \[
      |(0,1)| = \red{|\R| = |\R \times \R|} \blue{\;= |\R^{n \in \N}|}
    \]
  \end{theorem}

  \vspace{0.60cm}
  \begin{quote}
    \begin{center}
      ``Je le vois, mais je ne le crois pas !'' \\[4pt]
      \red{``I see it, but I don't believe it !''} \\[5pt]
      \hfill --- Cantor's letter to Dedekind (1877). 
    \end{center}
  \end{quote}

  \pause
  \begin{center}
    {\purple{\it $Q:$ Then, what is ``dimension''?}}
  \end{center}
\end{frame}
%%%%%%%%%%%%%%%