% func-def.tex

%%%%%%%%%%%%%%%
\begin{frame}{}
  \begin{center}
    {\LARGE Function}

    \pause
    \fig{width = 0.25\textwidth}{figs/aims}

    \pause
    \vspace{0.20cm}
    \centerline{\Large \red{PROOF! PROOF! PROOF!}}
  \end{center}
\end{frame}
%%%%%%%%%%%%%%%

%%%%%%%%%%%%%%%
\begin{frame}{}
  \begin{center}
    \teal{\LARGE Definition of Functions}
  \end{center}
\end{frame}
%%%%%%%%%%%%%%%

%%%%%%%%%%%%%%%
\begin{frame}{}
  \[
    R \subseteq A \times B
  \]

  \begin{center}
    is a \red{\it relation} from $A$ to $B$
  \end{center}
\end{frame}
%%%%%%%%%%%%%%%

%%%%%%%%%%%%%%%
\begin{frame}{}
  \begin{definition}[Function]
    $R \subseteq A \times B$ is a \red{\it function} from $A$ to $B$ if

    \[
      \red{\forall} a \in A: \red{\exists!} b \in B: (a, b) \in f.
    \]
  \end{definition}

  \pause
  \vspace{0.60cm}
  \begin{exampleblock}{Notations:}
    \[
      f: A \to B 
    \]

    \pause
    \[
      \dom{f} = A \qquad \ran{f} = f(A) \subseteq B \;\red{\triangleq}\; \cod{f}
    \]

    \pause
    \[
      f: a \mapsto b \;\red{\triangleq}\; f(a)
    \]
  \end{exampleblock}
\end{frame}
%%%%%%%%%%%%%%%

%%%%%%%%%%%%%%%
\begin{frame}{}
  \begin{definition}[Function]
    $R \subseteq A \times B$ is a \red{\it function} from $A$ to $B$ if

    \[
      \red{\forall} a \in A: \red{\exists!} b \in B: (a, b) \in f.
    \]
  \end{definition}

  \begin{alertblock}{For Proof:}
    \[
      \red{\forall a \in A:}
    \]
    \[
      \forall a \in A: \exists b \in B: (a, b) \in f
    \]

    \pause
    \[
      \red{\exists! b \in B:}
    \]
    \[
      \forall b, b' \in B: (a, b) \in f \land (a, b') \in f \implies b = b'
    \]
  \end{alertblock}
\end{frame}
%%%%%%%%%%%%%%%

%%%%%%%%%%%%%%%
\begin{frame}{}
  \[
    D: \real{} \to \real{}
  \]

  \[
    D(x) = \left\{\begin{array}{ll}
      1 & \text{if } x \in \mathbb{Q} \\
      0 & \text{if } x \in \real \setminus \mathbb{Q} 
    \end{array}\right.
  \]

  \vspace{0.60cm}
  \centerline{Dirichlet Function}
\end{frame}
%%%%%%%%%%%%%%%

%%%%%%%%%%%%%%%
% \begin{frame}{}
%   \fig{width = 0.50\textwidth}{figs/weierstrass-function}{\centerline{Weierstrass Function (1872)}}
% 
%   \[
%     f(x)=\sum_{n=0} ^\infty a^n \cos(b^n \pi x) 
%   \]
% 
%   \[
%     0 < a < 1,\; b \in 2\nat + 1,\; ab > 1+\frac{3}{2} \pi
%   \]
% \end{frame}
%%%%%%%%%%%%%%%

%%%%%%%%%%%%%%%
\begin{frame}{}
  \begin{exampleblock}{UD Problem 14.3 (g)}
    \[
      f: \mathbb{Q} \to \real{}
    \]

    \[
      f(x) = \left\{\begin{array}{ll}
	x + 1 & \text{if } x \in 2\integer{} \\
	x - 1 & \text{if } x \in 3\integer{} \\
	2     & \text{otherwise}
      \end{array}\right.
    \]
  \end{exampleblock}

  \pause
  \[
    x = 6
  \]
\end{frame}
%%%%%%%%%%%%%%%

%%%%%%%%%%%%%%%
\begin{frame}{}
  \begin{exampleblock}{UD Problem 14.5}
    \[
      f: \mathcal{P}(\real{}) \to \integer{}
    \]

    \[
      f(A) = \left\{\begin{array}{ll}
	\min(A \cap \nat{}) & \text{if } A \cap \nat{} \neq \emptyset \\
	-1 & \text{if } A \cap \nat{} = \emptyset
      \end{array}\right.
    \]
  \end{exampleblock}

  \pause
  \vspace{0.50cm}
  \begin{center}
    By the \red{\it Well-Ordering Principle} of $\nat{}$
  \end{center}
\end{frame}
%%%%%%%%%%%%%%%