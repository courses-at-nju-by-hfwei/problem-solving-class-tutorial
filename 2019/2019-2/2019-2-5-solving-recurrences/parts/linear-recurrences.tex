% linear-recurrences.tex

%%%%%%%%%%%%%%%
\begin{frame}{}
  \[
    \boxed{a_n = \green{f}(a_{n-1}, a_{n-2}, \ldots, a_{n-\red{t}}) + \blue{g(n)}}
  \]

  \pause
  \fig{width = 0.65\textwidth}{figs/aoa-classification-of-recurrences}
\end{frame}
%%%%%%%%%%%%%%%

%%%%%%%%%%%%%%%
% \begin{frame}{}
%   \fig{width = 0.60\textwidth}{figs/feedback}
% 
%   \vspace{0.30cm}
%   \centerline{\red{$Q:$ \large \it Recurrences without Base Cases?}}
% 
%   \pause
%   \vspace{0.30cm}
%   \begin{quote}
%     \centering
%     ``\teal{Boundary conditions} represent another class of details \\
%     that we typically \red{ignore}.''  \\[6pt]
% 
%     ``We shall generally \red{omit} statements of the boundary conditions of recurrences 
%     and \teal{assume that $T(n)$ is constant for small $n$}.'' \\[6pt]
% 
%     \hfill --- ``Technicalities in recurrences'', Chapter 4, CLRS
%   \end{quote}
% 
%   \pause
%   \vspace{0.30cm}
%   \centerline{\red{\large Not Exactly!}}
% \end{frame}
%%%%%%%%%%%%%%%

%%%%%%%%%%%%%%%
\begin{frame}{}
  \begin{theorem}[First-order Linear Recurrences with \red{Constant Coefficients}]
    \[
      T(n) = \red{r} T(n-1) + g(n) \quad \text{for } n > 0 \text{ with } T(0) = a
    \]

    \vspace{0.30cm}
    \[
      \teal{\boxed{T(n) = r^n a + \sum_{i=1}^{n} r^{n-i} g(i)}}
    \]
  \end{theorem}
\end{frame}
%%%%%%%%%%%%%%%

%%%%%%%%%%%%%%%
\begin{frame}{}
  \begin{theorem}[First-order Linear Recurrences]
    \[
      T(n) = \red{x_n} T(n-1) + y_n  \quad \text{ for } n > 0 \text{ with } T(0) = 0
    \]

    \[
      \teal{\boxed{T(n) = y_n + \sum_{1 \le j < n} y_j x_{j+1} x_{j+2} \cdots x_n}}
      % = \sum_{1 \le j \le n} \left( y_{j} \prod_{j+1 \le i \le n} x_i \right)
    \]
  \end{theorem}

  \pause
  \begin{align*}
    T(n) &= x_{n} T(n-1) + y_{n} \\
    &= \teal{x_{n} \left( x_{n-1} T(n-2) + y_{n-1} \right) + y_{n}} \\
    &= x_{n} x_{n-1} T(n-2) + x_{n} y_{n-1} + y_{n} \\
    &= \teal{x_{n} x_{n-1} \left( x_{n-2} T(n-3) + y_{n-2} \right) 
     + x_{n} y_{n-1} + y_{n}} \\
    &= x_{n} x_{n-1} x_{n-2} T(n-3) + x_{n} x_{n-1} y_{n-2} + x_{n} y_{n-1} + y_{n} \\
    &= \ldots
  \end{align*}
\end{frame}
%%%%%%%%%%%%%%%

%%%%%%%%%%%%%%%
\begin{frame}{}
  \begin{theorem}[First-order Linear Recurrences]
    \[
      T(n) = \red{x_n} T(n-1) + y_n  \quad \text{ for } n > 0 \text{ with } T(0) = 0
    \]

    \[
      \teal{\boxed{T(n) = y_n + \sum_{1 \le j < n} y_j x_{j+1} x_{j+2} \cdots x_n}} 
      % = \sum_{1 \le j \le n} \left( y_{j} \prod_{j+1 \le i \le n} x_i \right)
    \]
  \end{theorem}

  \pause
  \[
    \frac{T(n)}{\underbrace{x_n x_{n-1} \cdots x_1}_{\text{\red{summation factor}}}}
      = \frac{T(n-1)}{x_{n-1} \cdots x_1} + \frac{y_n}{x_n x_{n-1} \cdots x_1}
  \]

  \pause
  \[
    S(n) \triangleq \frac{T(n)}{x_n x_{n-1} \cdots x_1}
  \]

  \pause
  \[
    S(n) = S(n-1) + \frac{y_n}{x_n x_{n-1} \cdots x_1}
  \]
\end{frame}
%%%%%%%%%%%%%%%

%%%%%%%%%%%%%%%
\begin{frame}{}
  \[
    \boxed{T(n) = (1 + \frac{1}{n}) T(n-1) + 2  \quad \text{ for } n > 1 \text{ with } T(1) = 0}
  \]

  \pause
  \[
    \blue{x_n = 1 + \frac{1}{n} = \frac{n+1}{n}} \implies \pause x_n x_{n-1} \cdots x_1 = n + 1
  \]

  \pause
  \[
    \frac{T(n)}{n+1} = \frac{T(n-1)}{n} + \frac{2}{n+1} \quad \text{ for } n > 1
  \]

  \pause
  \vspace{0.30cm}
  \[
    \frac{T(n)}{n+1} = \frac{T(1)}{2} + 2 \sum_{3 \le k \le n+1} \frac{1}{k}
  \]

  \pause
  \vspace{0.30cm}
  \[
    \boxed{\teal{T(n) = 2(n+1) (H_{n+1} - \frac{3}{2})} \approx 2n \ln n - 1.846n}
  \]
\end{frame}
%%%%%%%%%%%%%%%

%%%%%%%%%%%%%%%
\begin{frame}{}
  \[
    \boxed{T(n) = (1 + \frac{1}{n}) T(n-1) + 2  \quad \text{ for } n > 1 \text{ with } T(1) = 0}
  \]

  \[
    \boxed{\teal{T(n) = 2(n+1) (H_{n+1} - \frac{3}{2})} \approx 2n \ln n - 1.846n}
  \]

  \pause
  \vspace{0.50cm}
  \[
    T(n) = (n + 1) + \frac{1}{n} \sum_{1 \le i \le n} \Big(T(i-1) + T(n-i) \Big)
  \]
  \[
    \text{ for } n > 1 \text{ with } T(0) = T(1) = 0
  \]

  \vspace{0.30cm}
  \begin{center}
    \red{average} number of comparisons of \textsc{Quicksort}
  \end{center}
\end{frame}
%%%%%%%%%%%%%%%

%%%%%%%%%%%%%%%
\begin{frame}{}
  \begin{exampleblock}{After-class Exercise}
    \[
      T(n) = T(n-1) - \frac{2T(n-1)}{n} + 2\left(1 - \frac{2T(n-1)}{n}\right), 
      n > 0 \text{ with } T(0) = 0
    \]
  \end{exampleblock}

  \vspace{0.30cm}
  \begin{center}
    \href{https://github.com/hengxin/problem-solving-class-paperswelove/blob/master/2nd-semester/Acta\%20Informatica\%20(Yao\%201978)\%20On\%20Random\%202-3\%20Trees.pdf}{\teal{``On Random 2-3 Trees'', Andrew C Yao, 1978}}
  \end{center}

  \pause
  \fig{width = 0.30\textwidth}{figs/doit-duck}
\end{frame}
%%%%%%%%%%%%%%%

%%%%%%%%%%%%%%%
\begin{frame}{}
  \begin{center}
    {\Large Higher-order Linear Recurrences}
  \end{center}

  \[
    a_n = x_1 a_{n-1} + x_2 a_{n-2} + \cdots + x_t a_{n-\red{t}} + g_{n} \quad \text{for } n \ge t
  \]
  \[
    \text{with } a_0, a_1, \cdots, a_{t-1}
  \]
\end{frame}
%%%%%%%%%%%%%%%

%%%%%%%%%%%%%%%
\begin{frame}{}
  \begin{theorem}[Linear \red{Homogeneous} Recurrences with \red{Constant} Coefficients]
    \[
      a_n = r_1 a_{n-1} + r_2 a_{n-2} + \cdots + r_t a_{n-t} \quad \text{for } n \ge t
    \]
    \[
      \text{with } a_0, a_1, \cdots, a_{t-1}
    \]

    \pause
    \vspace{-0.30cm}
    \[
      \teal{\text{``Characteristic polynomial''}: 
      \boxed{q(x) \;\red{\equiv}\; x^t - r_1 x^{t-1} - r_2 x^{t-2} - \cdots - r_t}}
    \]

    \pause
    \[
      \beta_1\; (m_1), \beta_2\; (m_2), \cdots, \beta_i\; (m_i), \cdots, \beta_{k}\; (m_k)
    \]
    \[
      \beta_{i} \text{ is a root with multiplicity } m_i \text{ and } m_1 + m_2 + \cdots + m_k = t
    \]

    \pause
    \vspace{-0.30cm}
    \[
      \red{\boxed{a_n = \sum_{0 \le j < m_1} c_{1j} n^j \beta_{1}^{n} 
        + \sum_{0 \le j < m_2} c_{2j} n^j \beta_{2}^{n} + \cdots 
        + \sum_{0 \le j < m_k} c_{kj} n^j \beta_{k}^{n}}}
    \]
    \begin{center}
      $a_n$ is a \red{linear combination} of $n^{j} \beta^{n}$ (called ``particular solutions'')
    \end{center}
  \end{theorem}
\end{frame}
%%%%%%%%%%%%%%%

%%%%%%%%%%%%%%%
\begin{frame}{}
  \[
    F_{n} = F_{n-1} + F_{n-2} \quad \text{for } n > 1 \text{ with } F_{0} = 0,\; F_{1} = 1
  \]

  \pause
  \[
    \teal{\boxed{x^2 - x - 1 = 0}} \implies 
    \phi = \frac{1 + \sqrt{5}}{2},\; \hat{\phi} = \frac{1 - \sqrt{5}}{2}
  \]

  \pause
  \[
    \red{\boxed{F_n = c_1 \phi^{n} + c_2 \hat{\phi}^{n}}}
  \]

  \pause
  \begin{align*}
    F_0 &= 0 = c_1 + c_2 \\
    F_1 &= 1 = c_1 \phi + c_2 \hat{\phi}
  \end{align*}

  \pause
  \[
    \red{\boxed{F_n = \frac{1}{\sqrt{5}} (\phi^{n} - {\hat{\phi}}^{n})}}
  \]
\end{frame}
%%%%%%%%%%%%%%%

%%%%%%%%%%%%%%%
\begin{frame}{}
  \[
    a_n = r_1 a_{n-1} + r_2 a_{n-2} + \cdots + r_t a_{n-t} \quad \text{for } n \ge t
  \]
  \[
    \teal{\boxed{q(x) \;\red{\equiv}\; x^t - r_1 x^{t-1} - r_2 x^{t-2} - \cdots - r_t}}
  \]
  \[
    \red{\boxed{a_n = \sum_{0 \le j < m_1} c_{1j} n^j \beta_{1}^{n} 
      + \sum_{0 \le j < m_2} c_{2j} n^j \beta_{2}^{n} + \cdots 
      + \sum_{0 \le j < m_k} c_{kj} n^j \beta_{k}^{n}}}
  \]

  \pause
  \begin{proof}
    \[
      \blue{\text{Take } \beta \; (m = 2) \text{ for example.}}
    \]

    \pause
    \[
      \beta^n = r_1 \beta^{n-1} + r_2 \beta^{n-2} + \cdots + r_t \beta^{n-t} \quad \text{for } n \ge t
    \]
    \pause
    \vspace{-0.50cm}
    \[
      \beta^{n-t} \red{q(\beta)} = 0
    \]

    \pause
    \vspace{-0.50cm}
    \[
      n\beta^{n} = r_1 (n-1) \beta^{n-1} + r_2 (n-2) \beta^{n-2} + \cdots + r_t (n-t) \beta^{n-t} \quad \text{for } n \ge t
    \]
    \pause
    \vspace{-0.50cm}
    \[
      \beta^{n-t} \left(\left(n-t\right)\red{q(\beta)} + \beta \red{q'(\beta)}\right) = 0
    \]
  \end{proof}
\end{frame}
%%%%%%%%%%%%%%%

%%%%%%%%%%%%%%%
\begin{frame}{}
  \[
    a_n = r_1 a_{n-1} + r_2 a_{n-2} + \cdots + r_t a_{n-t} \quad \text{for } n \ge t
  \]
  \[
    \teal{\boxed{q(x) \;\red{\equiv}\; x^t - r_1 x^{t-1} - r_2 x^{t-2} - \cdots - r_t}}
  \]
  \[
    \red{\boxed{a_n = \sum_{0 \le j < m_1} c_{1j} n^j \beta_{1}^{n} 
      + \sum_{0 \le j < m_2} c_{2j} n^j \beta_{2}^{n} + \cdots 
      + \sum_{0 \le j < m_k} c_{kj} n^j \beta_{k}^{n}}}
  \]

  \begin{proof}[Proof Cont.]
    \begin{center}
      $a_n$ is a \red{linear combination} of $n^{j} \beta^{n}$
      \\[15pt] \pause

      We also need to prove that there are \red{\it no} other solutions.
    \end{center}

    \pause
    \fig{width = 0.20\textwidth}{figs/omitted}
  \end{proof}
\end{frame}
%%%%%%%%%%%%%%%

%%%%%%%%%%%%%%%
\begin{frame}{}
  \[
    a_n = 5a_{n-1} - 6a_{n-2} \quad \text{for } n \ge 2 \text{ with } a_0 = 0, a_1 = 1
  \]

  \pause
  \[
    \teal{\boxed{x^2 - 5x + 6 = (x-2) (x-3) = 0}} \implies x = 2,\; 3
  \]

  \pause
  \[
    \red{\boxed{a_n = c_1 2^n + c_2 3^n}}
  \]
  
  \pause
  \begin{align*}
    a_0 &= 0 = c_1 + c_2 \\
    a_1 &= 1 = 2c_1 + 3c_2
  \end{align*}

  \pause
  \[
    \red{\boxed{a_n = 3^n - 2^n}}
  \]
\end{frame}
%%%%%%%%%%%%%%%

%%%%%%%%%%%%%%%
\begin{frame}{}
  \[
    a_n = 5 a_{n-1} - 8 a_{n-2} + 4 a_{n-3} 
    \quad \text{for } n \ge 3 \text{ with } a_0 = 0,\; a_1 = 1,\; a_2 = 4
  \]

  \pause
  \[
    \teal{\boxed{x^3 - 5x^2 + 8x - 4 = 0}}
  \]

  \pause
  \[
    (x-1) (x-2)^2 = 0 \implies x_1 = 1,\; x_{2} = 2,\; x'_{2} = 2
  \]

  \pause
  \[
    a_n = c_1 \cdot 1^{n} + c_{2} \cdot \red{2^{n}} + c'_{2} \cdot \red{n 2^{n}}
  \]

  \pause
  \vspace{-0.30cm}
  \begin{align*}
    a_0 &= 0 = c_1 + c_2 \\
    a_1 &= 1 = c_1 + 2c_2 + 2c'_{2} \\
    a_2 &= 4 = c_1 + 4c_2 + 8c'_{2}
  \end{align*}

  \pause
  \[
    \red{\boxed{a_n = n 2^{n-1}}}
  \]
\end{frame}
%%%%%%%%%%%%%%%

%%%%%%%%%%%%%%%
\begin{frame}{}
  \[
    a_n = 2a_{n-1} - a_{n-2} + 2a_{n-3} 
    \quad \text{for } n \ge 3 \text{ with } a_0 = 1,\; a_1 = 0,\; a_2 = -1
  \]

  \pause
  \[
    \teal{\boxed{\teal{x^3 - 2x^2 + x - 2}} = (x^2 + 1) (x-2) = 0}
    \implies x = 2,\; \red{i},\; \red{-i}
  \]

  \pause
  \[
    \red{\boxed{a_n = c_1 2^n + c_2 i^n + c_3 (-i)^n}}
  \]

  \pause
  \vspace{-0.50cm}
  \begin{align*}
    a_0 &= 1 = c_1 + c_2 + c_3 \\
    a_1 &= 0 = 2c_1 + c_2 i - c_3 i \\
    a_2 &= -1 = 4c_1 - c_2 - c_3
  \end{align*}

  \pause
  \vspace{-0.30cm}
  \[
    \red{\boxed{a_n = \frac{1}{2} i^{n} \left(1 + (-1)^n \right)}}
  \]

  \pause
  \[
    \blue{1},\; \red{0},\; \blue{-1},\; \red{0},\; \blue{1},\; \red{0},\; \blue{-1},\; \red{0},\;
    \blue{1},\; \red{0},\; \blue{-1},\; \red{0} \cdots
  \]
\end{frame}
%%%%%%%%%%%%%%%

%%%%%%%%%%%%%%%
\begin{frame}{}
  \[
    a_n = 2a_{n-1} - a_{n-2} + 2a_{n-3} 
    \quad \text{for } n \ge 3 \text{ with } \red{a_0,\; a_1,\; a_2}
  \]

  \[
    \red{\boxed{a_n = c_1 2^n + c_2 i^n + c_3 (-i)^n}}
  \]

  \vspace{-0.50cm}
  \begin{align*}
    a_0 &= c_1 + c_2 + c_3 \\
    a_1 &= 2c_1 + c_2 i - c_3 i \\
    a_2 &= 4c_1 - c_2 - c_3
  \end{align*}

  \pause
  \[
    a_0 = 1,\; a_1 = 0,\; a_2 = -1 \implies a_n = \frac{1}{2} i^{n} \left(1 + (-1)^n \right)
  \]

  \pause
  \[
    a_0 = 1,\; a_1 = 2,\; a_2 = 4 \implies a_{n} = 2^n
  \]

  \pause
  \begin{center}
    \red{Pay attention to initial conditions in linear recurrences!}
  \end{center}
\end{frame}
%%%%%%%%%%%%%%%

%%%%%%%%%%%%%%%
% \begin{frame}{}
%   \[
%     a_{n} = 2a_{n-1} - a_{n-2} \quad \text{for } n \ge 2 \text{ with } a_{0} = 1,\; a_{1} = 2
%   \]
% 
%   \pause
%   \[
%     \teal{x^2 - 2x + 1 = (x - 1)^2} \implies x = 1,\; x' = 1
%   \]
% 
%   \pause
%   \[
%     a_{n} = c_{0} 1^{n} + c_{1} n 1^{n}
%   \]
% \end{frame}
%%%%%%%%%%%%%%%

%%%%%%%%%%%%%%%
\begin{frame}{}
  \begin{exampleblock}{Additional Problem}
    To give initial conditions $a_0, a_1$, and $a_2$ such that the growth rate of the solution to
    \[
      a_n = 2a_{n-1} - a_{n-2} + 2a_{n-3},\; n > 2
    \]
    is \blue{\it (1)} {\it constant}; \blue{\it (2)} {\it exponential}; \blue{\it (3)} {\it fluctuating in sign}.
  \end{exampleblock}

  \pause
  \vspace{0.50cm}
  \fig{width = 0.25\textwidth}{figs/doit-duck}
\end{frame}
%%%%%%%%%%%%%%%

%%%%%%%%%%%%%%%
\begin{frame}{}
  \begin{center}
    {\large First-order Linear \red{Non-homogeneous} Recurrences}
  \end{center}

  \[
    a_n = r_1 a_{n-1} + r_2 a_{n-2} + \cdots + r_t a_{n-t} + \red{r} \quad \text{for } n \ge t
  \]

  % \pause
  % \fig{width = 0.60\textwidth}{figs/fib-noip}
  % \[
  %   T(n) = T(n-1) + T(n-2) + 2 \quad \text{for } n > 2 \text{ with } T(1) = T(2) = 0
  % \]

  % \pause
  % \vspace{-0.30cm}
  % \[
  %   \boxed{T(n,k) = \left\{\begin{array}{lr}
  %     0, & k = 0 \lor n = k \\
  %     T(n-1, k) + T(n-1, k-1) + c, & \text{\it o.w.}
  %   \end{array}\right.}
  % \]
\end{frame}
%%%%%%%%%%%%%%%

%%%%%%%%%%%%%%%
% \begin{frame}{}
%   \[
%     a_n = 5a_{n-1} - 6a_{n-2} + 2, \; n \ge 2 \; (a_0 = 0, a_1 = 1)
%   \]
% 
%   \pause
%   \[
%     \boxed{\teal{a_n = c_0 3^n - c_1 2^n + \red{c_2}}}
%   \]
% 
%   \pause
%   \[
%     \boxed{\teal{c_2 = 5c_2 - 6c_2 + 2}} \implies c_2 = 1
%   \]
% 
%   \pause
%   \begin{align*}
%     a_0 &= 0 = c_0 + c_1 + 1\\
%     a_1 &= 1 = 3c_0 + 2c_1 + 1
%   \end{align*}
% 
%   \pause
%   \[
%     \red{\boxed{a_n = 2 \cdot 3^n - 3 \cdot 2^n + 1}}
%   \]
% \end{frame}
%%%%%%%%%%%%%%%

%%%%%%%%%%%%%%%
\begin{frame}{}
  \[
    a_n = 5a_{n-1} - 6a_{n-2} + 2 \quad \text{for } n \ge 2 \text{ with } a_0 = 0,\; a_1 = 1
  \]

  \pause
  \[
    a_n - 5a_{n-1} + 6a_{n-2} - 2 = 0 \quad \text{for } n \ge 2 \text{ with } a_0 = 0,\; a_1 = 1
  \]

  \pause
  \[
    a_{n-1} - 5a_{n-2} + 6a_{n-3} - 2 = 0 \quad \text{for } \red{n \ge 3 \text{ with } a_2 = 7}
  \]

  \pause
  \[
    a_n - 6a_{n-1} + 11a_{n-2} - 6a_{n-3} = 0
  \]

  \pause
  \[
    \boxed{\teal{x^3 - 6x^2 + 11x - 6} = (x-1) (x-2) (x-3) \implies x_1 = 1,\; x_2 = 2,\; x_3 = 3}
  \]

  \pause
  \[
    \red{\boxed{a_n = c_1 1^n + c_2 2^n + c_3 3^n}}
  \]

  \pause
  \[
    \red{\boxed{a_n = 2 \cdot 3^n - 3 \cdot 2^n + 1}}
  \]
\end{frame}
%%%%%%%%%%%%%%%