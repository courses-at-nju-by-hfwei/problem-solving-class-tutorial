% overview.tex

%%%%%%%%%%%%%%%
\begin{frame}{}
  \fig{width = 0.40\textwidth}{figs/AoA}
  \begin{center}
    \red{\large The Analysis of Algorithms}
  \end{center}
\end{frame}
%%%%%%%%%%%%%%%

%%%%%%%%%%%%%%%
\begin{frame}{}
  \fig{width = 0.50\textwidth}{figs/knuth-on-chair}

  \begin{center}
    \teal{Donald E. Knuth (1938 $\sim$)}
  \end{center}
\end{frame}
%%%%%%%%%%%%%%%

%%%%%%%%%%%%%%%
\begin{frame}{}
  \fig{width = 0.40\textwidth}{figs/logo-turing-award}

  \begin{center}
    \teal{Donald E. Knuth (1974)}
  \end{center}

  \pause
  \begin{quote}
    \begin{center}
      {\large
      ``For his major contributions to \red{ the analysis of algorithms} \\[5pt]
      and \blue{the design of programming languages}, \\[5pt]
      and in particular for his contributions to \\[5pt]
      the \purple{``art of computer programming''} through \\[5pt]
      his well-known books in a continuous series by this title.''}
    \end{center}
  \end{quote}
\end{frame}
%%%%%%%%%%%%%%%

%%%%%%%%%%%%%%%
\begin{frame}{}
  \begin{center}
    \only<3-4>{
      \href{https://www.cut-the-knot.org/blue/LamesTheorem.shtml}{\teal{Fibonacci numbers in the analysis of Euclid's \textsc{gcd} algorithm}} \\[6pt]
    }
    \only<4>{
      \href{https://youtu.be/jmcSzzN1gkc}{\teal{$H_n$ in the analysis of \textsc{find-max} @ Stanford Lecture by Knuth}}
    }
  \end{center}

  \vspace{0.30cm}
  \begin{center}
    \begin{quote}
      {\large 
      ``People who \red{analyze algorithms} have \blue{double happiness.} \\[8pt] \pause 
      First of all they experience the sheer beauty of elegant \red{mathematical patterns} 
      that surround elegant \blue{computational procedures}. \\[6pt]
      \uncover<5->{
      Then they receive a \red{practical payoff} 
      when their theories make it possible to get other jobs done 
      \blue{more quickly and more economically}.''}
      }
    \end{quote}
  \end{center}
\end{frame}
%%%%%%%%%%%%%%%

%%%%%%%%%%%%%%%
\begin{frame}{}
  \begin{center}
    \red{How Fast is It?}
  \end{center}

  \fig{width = 0.50\textwidth}{figs/fast}

  \pause
  \begin{center}
    Time \textcolor{lightgray}{(and Space)} Complexity of Algorithms
  \end{center}

  \pause
  \[
    O \quad \Omega \quad \Theta
  \]
  \[
    o \quad \omega
  \]
\end{frame}
%%%%%%%%%%%%%%%

%%%%%%%%%%%%%%%
\begin{frame}{}
  \begin{center}
    \red{Space Complexity of Algorithms} \\[15pt] \pause
    We only care about the \red{extra} space caused by the algorithm.  \\[8pt] \pause
    The space for \red{inputs} is not part of space complexity of algorithms.
  \end{center}

  \pause
  \[
    \Call{insertion-sort}{A, n}: O(1) \quad \teal{(\text{constant})}
  \]
\end{frame}
%%%%%%%%%%%%%%%

%%%%%%%%%%%%%%%
\begin{frame}{}
  \begin{center}
    \red{Is it the Fastest?}
  \end{center}

  \fig{width = 0.50\textwidth}{figs/bolt}

  \pause
  \begin{center}
    Complexity of Problems  \\[8pt] \pause
    This is much harder and is not our focus today.
  \end{center}
\end{frame}
%%%%%%%%%%%%%%%

%%%%%%%%%%%%%%%
\begin{frame}{}
  \fig{width = 0.40\textwidth}{figs/problem-solution}

  \pause
  \begin{center}
    Whenever you design an algorithm, \\[6pt] \pause
    you provide an \red{upper bound} for the \blue{complexity of the problem}.  \\[15pt] \pause
    Whenever you encounter a ``hardcore'' of the problem, \\[6pt] \pause
    you obtain a \red{lower bound} for \blue{all possible algorithms}.  \\[20pt] \pause

    Often, there is an \purple{``algorithmic gap''} between them. \\[8pt] \pause
    When the gap is gone, you get the \red{optimal} algorithm.
  \end{center}

  \pause
  \[
    \textsf{sorting}(A, n): \Theta(n \log n) = O(n \log n) \cap \Omega(n \log n)
  \]
\end{frame}
%%%%%%%%%%%%%%%