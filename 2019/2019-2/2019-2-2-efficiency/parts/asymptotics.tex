% asymptotics.tex

%%%%%%%%%%%%%%%
\begin{frame}{}
  \begin{center}
    \red{$Q:$ How fast is your algorithm?}  \\[15pt] 

    \pause
    \vspace{-0.50cm}
    \fig{width = 0.40\textwidth}{figs/listen-carefully}
    \vspace{-1.50cm}

    \pause
    \[
        W(n) = \frac{c_5 + c_6 + c_7}{2} n^2 
            + (c_1 + c_2 + c_4 + c_8 - \frac{c_5 + c_6 + c_7}{2}) n
            - (c_2 + c_4 + c_5 + c_8)
    \]
  \end{center}
\end{frame}
%%%%%%%%%%%%%%%

%%%%%%%%%%%%%%%
\begin{frame}{}
  \begin{columns}
    \column{0.60\textwidth}
      \fig{width = 0.95\textwidth}{figs/big-oh-knuth-paper}
    \column{0.40\textwidth}
      \fig{width = 0.85\textwidth}{figs/knuth-on-chair}
  \end{columns}

  \vspace{0.50cm}
  \begin{alertblock}{Reference:}
    {\it ``Big Omicron and Big Omega and Big Theta''}, Donald E. Knuth, 1976.
  \end{alertblock}

  \pause
  \begin{center}
    \red{\large Asymptotics}
  \end{center}
\end{frame}
%%%%%%%%%%%%%%%

%%%%%%%%%%%%%%%
\begin{frame}{}
  \begin{center}
    \red{$Q:$ How fast is your algorithm?}  \\[15pt] 

    \[
        W(n) = \frac{c_5 + c_6 + c_7}{2} n^2 
            + (c_1 + c_2 + c_4 + c_8 - \frac{c_5 + c_6 + c_7}{2}) n
            - (c_2 + c_4 + c_5 + c_8)
    \]

    \pause
    \[
      \red{\boxed{W(n) = O(n^2)}}
    \]

    \pause
    ``Order at most $n^2$''  \\[15pt]

    \pause
    ``$W(n)$ is a function whose \red{order of magnitude} is \teal{upper-bounded} \\
    by a \blue{constant times $n^2$}, \purple{for all large $n$}.''
  \end{center}
\end{frame}
%%%%%%%%%%%%%%%

%%%%%%%%%%%%%%%
\begin{frame}{}
  \begin{center}
    \[
      \boxed{f(n) = O(g(n))}
    \]

    \pause
    ``$f(n)$ is a function whose \red{order of magnitude} is \teal{upper-bounded} \\
    by a \blue{constant times $g(n)$}, \purple{for all large $n$}.''
    \pause
    \vspace{0.30cm}
    \fig{width = 0.35\textwidth}{figs/big-oh}

    \pause
    \vspace{-0.50cm}
    \[
        O(g(n)) = \Big\{f(n) \mid \blue{\exists c > 0}, 
        \purple{\exists n_0 > 0, \forall n \ge n_0}: 
        0 \le f(n) \;\teal{\le}\; \blue{c g(n)} \Big\}
    \]
  \end{center}
\end{frame}
%%%%%%%%%%%%%%%

%%%%%%%%%%%%%%%
\begin{frame}{}
  \[
   \boxed{f(n) = O(g(n))}
  \]

  \[
      O(g(n)) = \Big\{f(n) \mid \blue{\exists c > 0}, 
      \purple{\exists n_0 > 0, \forall n \ge n_0}: 
      0 \le f(n) \;\teal{\le}\; \blue{c g(n)} \Big\}
  \]

  \pause
  \[
	\purple{\Big\{ \quad \Big\}}
  \]

  \pause
  \vspace{0.50cm}
  \begin{center}
    It is a tradition to write $f(n) = O(g(n))$ instead of $f(n) \in O(g(n))$.
  \end{center}
\end{frame}
%%%%%%%%%%%%%%%

%%%%%%%%%%%%%%%
\begin{frame}{}
  \[
    42n^2 + 2020n \;\blue{= O(n^2)} \pause \;\red{= O(n^3)}
  \]

  \pause
  \[
    42n^2 + 2020n \in \;\blue{O(n^2)} \;\red{\subseteq O(n^3)}
  \]
\end{frame}
%%%%%%%%%%%%%%%

%%%%%%%%%%%%%%%
\begin{frame}{}
  \[
    O\big(f(n)\big) + O\big(g(n)\big) \triangleq \pause
    \Big\{ h + l \mid h \in O\big(f(n)\big), l \in O\big(g(n)\big) \Big\}
  \]

  \pause
  \[
    O\big(f(n)\big) O\big(g(n)\big) \triangleq 
    \Big\{ h l \mid h \in O\big(f(n)\big), l \in O\big(g(n)\big) \Big\}
  \]

  \pause
  \[
    \textcolor{lightgray}{O\big(f(n)\big) - O\big(g(n)\big) \triangleq}
  \]
\end{frame}
%%%%%%%%%%%%%%%

%%%%%%%%%%%%%%%
\begin{frame}{}
  \[
      O(g(n)) = \Big\{f(n) \mid \blue{\exists c > 0}, 
      \purple{\exists n_0 > 0, \forall n \ge n_0}: 
      0 \le f(n) \;\teal{\le}\; \blue{c g(n)} \Big\}
  \]

  \pause
  \[
    42n = O(0.50 n^2) \pause \qquad 42n^2 = O(0.50 n^2)
  \]

  \pause
  \begin{center}
    \red{$Q:$ What does $O(1)$ mean?} \\[15pt] \pause
    \blue{$A:$ It means constants.}
  \end{center}
\end{frame}
%%%%%%%%%%%%%%%

%%%%%%%%%%%%%%%
\begin{frame}{}
  \[
    \Omega(g(n)) = \Big\{f(n) \mid \blue{\exists c > 0}, 
    \purple{\exists n_0 > 0, \forall n \ge n_0}: 
    0 \le \blue{c g(n)} \;\teal{\le}\; f(n) \Big\}
  \]

  \pause
  \[
    0.50 n^2 = \Omega(42 n) \pause \qquad 0.50 n^2 = \Omega(42n^2)
  \]

  \pause
  \begin{align*}
    \Theta(g(n)) = \Big\{f(n) \mid\; &\blue{\exists c_1 > 0, \exists c_2 > 0}, \purple{\exists n_0 > 0, \forall n \ge n_0}: \\ 
    & 0 \le \blue{c_1 g(n)} \;\teal{\le}\; f(n) \;\teal{\le}\; \blue{c_2 g(n)} \Big\}
  \end{align*}

  \pause
  \[
    0.50 n^2 = \Theta(42n^2)
  \]
\end{frame}
%%%%%%%%%%%%%%%

%%%%%%%%%%%%%%%
\begin{frame}{}
  \[
	o(g(n)) = \Big\{f(n) \mid \textcolor{red}{\forall c > 0}, \exists n_0 > 0, \forall n \ge n_0: 0 \le f(n) \;\red{<}\; c g(n)\Big\}
  \]

  \pause
  \[
    42n = o(0.50 n^2)
  \]

  \pause
  \[
	\omega(g(n)) = \Big\{f(n) \mid \textcolor{red}{\forall c > 0}, \exists n_0 > 0, \forall n \ge n_0: 0 \le c g(n) \;\red{<}\; f(n)\Big\}
  \]

  \pause
  \[
    0.50 n^2 = \omega(42n)
  \]
\end{frame}
%%%%%%%%%%%%%%%

%%%%%%%%%%%%%%%
\begin{frame}{}
  \[
    O \quad \Omega \quad \Theta
  \]
  \[
    o \quad \omega \quad \teal{\theta}
  \]

  \pause
  \vspace{0.60cm}
  \[
	\teal{f(n) \;\red{\sim}\; g(n) \iff \lim_{n \to \infty} \frac{f(n)}{g(n)} = 1}
  \]

  \pause
  \vspace{0.60cm}
  \[
    42 n^2 + 2020 n \sim 42 n^2 + 2019 n
  \]
\end{frame}
%%%%%%%%%%%%%%%

%%%%%%%%%%%%%%%
\begin{frame}{}
  \[
    f(n) = \Theta(g(n)) \iff f(n) = O(g(n)) \land f(n) = \Omega(g(n))
  \]

  \begin{align*}
    f(n) = O(g(n)) &\iff g(n) = \Omega(f(n)) \\[6pt]
    f(n) = o(g(n)) &\iff g(n) = \omega(f(n))
  \end{align*}
\end{frame}
%%%%%%%%%%%%%%%

%%%%%%%%%%%%%%%
\begin{frame}{}
  \[
    O\big(f(n)\big) + O\big(g(n)\big) = O\big(f(n) + g(n)\big)
  \]

  \pause
  \[
    O\big(f(n)\big) O\big(g(n)\big) = O\big(f(n) g(n)\big)
  \]
\end{frame}
%%%%%%%%%%%%%%%

%%%%%%%%%%%%%%%
\begin{frame}{}
  \begin{center}
    \red{$Q:$ How to compare functions in terms of $O/\Omega/\Theta$?}
  \end{center}

  \pause
  \begin{align*}
    O(1) &= O(\log \log n) = O(\log n) = O((\log n)^{c}) \\[6pt]
    &= O(n^{\epsilon}) = O(n^c) \\[6pt]
    &= O(n^{c} \log n) = O(n^{\log n}) = O(c^n) = O(n^n)
  \end{align*}
  \[
    \blue{(0 < \epsilon < 1 < c)}
  \]
\end{frame}
%%%%%%%%%%%%%%%

%%%%%%%%%%%%%%%
\begin{frame}{}
  \fig{width = 0.60\textwidth}{figs/growth-order}
\end{frame}
%%%%%%%%%%%%%%%

%%%%%%%%%%%%%%%
\begin{frame}{}
  \begin{alertblock}{Stirling Formula (by {\it James Stirling}):}
    \begin{columns}
      \column{0.50\textwidth}
        \[
        n! \;\red{\sim}\; \sqrt{2 \pi n} \Big(\frac{n}{e}\Big)^{n}
        \]
      \column{0.50\textwidth}
        \fig{width = 0.30\textwidth}{figs/stirling-formula-wiki-qrcode}
    \end{columns}
  \end{alertblock}

  \pause
  \[
    \log(n!) = \Theta(n \log n)
  \]

  \pause
  \[
    H_n = \sum_{k=1}^{n} \frac{1}{k} = \Theta(\log n)
  \]
\end{frame}
%%%%%%%%%%%%%%%