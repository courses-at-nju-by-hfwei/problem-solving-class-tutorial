% complexity.tex

%%%%%%%%%%%%%%%
\begin{frame}{}
  \begin{center}
    \red{$Q:$ How fast is your algorithm?}  \\[15pt] 
    \pause
    \blue{$A:$ It runs 3.1415926 seconds.}
  \end{center}

  \vspace{0.50cm}
  \fig{width = 0.60\textwidth}{figs/tle-oj}
\end{frame}
%%%%%%%%%%%%%%%

%%%%%%%%%%%%%%%
\begin{frame}{}
  \begin{columns}
    \column{0.25\textwidth}
    \column{0.50\textwidth}
      \red{Disadvantages:} \\[10pt]  \pause
      \begin{itemize}[<+->]
        \setlength{\itemsep}{6pt}
        \item On different machines
        \item At different time
        \item On different inputs
      \end{itemize}
    \column{0.25\textwidth}
  \end{columns}

  \pause
  \vspace{0.60cm}
  \begin{center}
    \red{No Standards.}
  \end{center}
\end{frame}
%%%%%%%%%%%%%%%

%%%%%%%%%%%%%%%
\begin{frame}{}
  \begin{center}
    We need a uniform \red{model of computation}. \\[8pt] \pause
  \end{center}

  \pause
  \begin{center}
    \href{https://www8.cs.umu.se/kurser/TDBA77/VT06/algorithms/BOOK/BOOK/NODE12.HTM}{\blue{The RAM (Random Access Machine) Model of Computation}}
  \end{center}

  \fig{width = 0.70\textwidth}{figs/ram-model}
\end{frame}
%%%%%%%%%%%%%%%

%%%%%%%%%%%%%%%
\begin{frame}{}
  \begin{center}
    \href{https://www8.cs.umu.se/kurser/TDBA77/VT06/algorithms/BOOK/BOOK/NODE12.HTM}{\blue{The RAM (Random Access Machine) Model of Computation}}
  \end{center}

  \begin{columns}
    \column{0.10\textwidth}
    \column{0.80\textwidth}
      \begin{itemize}
        \setlength{\itemsep}{6pt}
        \item Each memory access takes constant time.
        \item Each \red{\it ``primitive''} operation takes constant time.
        \item Compound operations should be decomposed.
      \end{itemize}
    \column{0.10\textwidth}
  \end{columns}

  \pause
  \vspace{0.60cm}
  \begin{center}
    \red{Counting up the number of time units.}
  \end{center}
\end{frame}
%%%%%%%%%%%%%%%

%%%%%%%%%%%%%%%
\begin{frame}{}
  \begin{columns}
    \column{0.25\textwidth}
    \column{0.50\textwidth}
      \red{Disadvantages:}
      \begin{itemize}
        \setlength{\itemsep}{6pt}
        \item On different machines
        \item At different time
        \item \red{On different inputs}
      \end{itemize}
    \column{0.25\textwidth}
  \end{columns}

  \pause
  \vspace{0.60cm}
  \begin{center}
    \red{Counting up the number of time units} \\[5pt]
    \blue{as a function of the input size} \\[5pt]
    \purple{in typical cases.}
  \end{center}
\end{frame}
%%%%%%%%%%%%%%%

%%%%%%%%%%%%%%%
\begin{frame}{}
  \fig{width = 0.80\textwidth}{figs/insertion-sort-time}

  \pause
  \vspace{-0.30cm}
  \begin{align*}
    T(n) = c_1 n &+ c_2 (n-1) + c_4 (n-1) \\
    &+ c_5 \sum_{j=2}^{n} t_j + c_6 \sum_{j=2}^{n} (t_j - 1) + c_7 \sum_{j=2}^{n} (t_j - 1) + c_8 (n-1)
  \end{align*}

  \begin{center}
    \blue{$\ldots$ as a function of the input size $\ldots$}
  \end{center}
\end{frame}
%%%%%%%%%%%%%%%
  
%%%%%%%%%%%%%%%
\begin{frame}
  \fig{width = 0.80\textwidth}{figs/insertion-sort-time}

  \vspace{-0.30cm}
  \begin{align*}
    T(n) = c_1 n &+ c_2 (n-1) + c_4 (n-1) \\
    &+ c_5 \sum_{j=2}^{n} \red{t_j} + c_6 \sum_{j=2}^{n} (\red{t_j} - 1) + c_7 \sum_{j=2}^{n} (\red{t_j} - 1) + c_8 (n-1)
  \end{align*}

  \pause
  \begin{center}
    \red{$T(n)$: Depends on {\it which} input of size $n$}
  \end{center}
\end{frame}
%%%%%%%%%%%%%%%

%%%%%%%%%%%%%%%
\begin{frame}{}
  \begin{center}
    \purple{$\ldots$ in typical cases.}
  \end{center}

  \pause
  \vspace{0.50cm}
  \begin{center}
    {\large \teal{Problem $P$ \qquad Algorithm $A$}} \\[8pt] \pause
    {\large \red{Inputs: $\mathcal{X}_{n}$ of size $n$}}
  \end{center}

  \pause
  \[
    W(n) = \max_{x \in \mathcal{X}_{n}} T(x)
  \]

  \pause
  \[
    B(n) = \min_{x \in \mathcal{X}_{n}} T(x)
  \]

  \pause
  \[
	A(n) = \boxed{\red{\sum_{x \in \mathcal{X}_{n}} T(x) \cdot P(x)}} \pause 
	     = \blue{\mathbb{E} [T]} \pause 
		 = \boxed{\red{\sum_{t \in T(\mathcal{X}_{n})} t \cdot P(T = t)}}
  \]
\end{frame}
%%%%%%%%%%%%%%%

%%%%%%%%%%%%%%%
\begin{frame}{}
  \fig{width = 0.80\textwidth}{figs/insertion-sort-time}

  \pause
  \[
    B(n) = (c_1 + c_2 + c_4 + c_5 + c_8) n - (c_2 + c_4 + c_5 + c_8)
  \]

  \pause
  \[
    W(n) = \frac{c_5 + c_6 + c_7}{2} n^2 
        + (c_1 + c_2 + c_4 + c_8 - \frac{c_5 + c_6 + c_7}{2}) n
        - (c_2 + c_4 + c_5 + c_8)
  \]

  \pause
  \[
    A(n) = \pause \textcolor{lightgray}{2.25 n^2 + 7.75 n - 3 \red{H_n} - 6 \qquad (H_n = \sum_{k=1}^{n} \frac{1}{k}\approx \ln n})
  \]
\end{frame}
%%%%%%%%%%%%%%%