% correctness.tex

%%%%%%%%%%%%%%%
\begin{frame}{}
  \begin{quote}
    \begin{center}
    {\large
      What We Talk About \\[8pt]
      When We Talk About \blue{Correctness of Algorithms} $\cdots$
    }
    \end{center}
  \end{quote}
  % \fig{width = 0.30\textwidth}{figs/what-we-talk-about-love}
  
  \pause
  \vspace{0.30cm}
  \begin{center}
    \red{What is the correctness of algorithms?}
  \end{center}
\end{frame}
%%%%%%%%%%%%%%%

%%%%%%%%%%%%%%%
\begin{frame}{}
  \begin{definition}[Correctness of Algorithms \textcolor{lightgray}{(Not A Formal Definition)}]
    An algorithm is \teal{considered} correct if it meets its \red{specification}.
  \end{definition}
  
  \vspace{0.80cm}
  \begin{columns}
    \column{0.50\textwidth}
      \pause
      \begin{center}
        Alg: $B \gets \texttt{Sort}(A)$ \\[8pt] \pause
        \red{Spec: $B$ is sorted} \pause
        \[
          \forall 0 \le i < n-1: B[i] \le B[i + 1]
        \]
      \end{center}
    \column{0.50\textwidth}
      \pause
      \begin{center}
        Alg: $d \gets \texttt{Euclid}(a, b)$ \pause
        \[
          \red{\text{Spec}: d = \textsf{gcd}(a, b)}
        \]
      \end{center}
  \end{columns}
  
  \pause
  \vspace{0.50cm}
  \begin{center}
    We use \blue{specification languages} to specify specs. \\[8pt] \pause
    Generally, they are \teal{mathematical logic + theory of domain knowledge}.
  \end{center}
\end{frame}
%%%%%%%%%%%%%%%

%%%%%%%%%%%%%%%
\begin{frame}{}
  \begin{center}
    Correctness: Partial Correctness \& Total Correctness
  \end{center}
  
  \pause
  \begin{definition}[Total Correctness]
    Total Correctness = Partial Correctness + Termination 
  \end{definition}
  
  \pause
  \vspace{0.50cm}
  \begin{definition}[Partial Correctness]
    \red{\it If} the algorithm terminates, it meets its specification.
  \end{definition}
  
  \pause
  \vspace{0.50cm}
  \begin{definition}[Total Correctness (Revisited)]
    The algorithm indeed terminates \red{and} it meets its specification.
  \end{definition}
\end{frame}
%%%%%%%%%%%%%%%

%%%%%%%%%%%%%%%
\begin{frame}{}
  \begin{center}
    \blue{Separating ``partial correctness'' from ``termination''}
  \end{center}
  
  \begin{definition}[Total Correctness]
    Total Correctness = Partial Correctness + Termination 
  \end{definition}
  
  \vspace{0.50cm}
  \begin{itemize}
    \setlength{\itemsep}{8pt}
    \pause
    \item They are intrinsically different for serious theoretical reasons.
    \pause
    \item Different \red{proof methods} for them \red{(Hoare Logic)}
      \begin{itemize}
        \setlength{\itemsep}{5pt}
        \item \teal{(Loop) Invariants for ``partial correctness''}
        \item \teal{Variants for ``termination''}
      \end{itemize}
    \pause
    \item ``Termination'' is often much easier for sequential algorithms.
  \end{itemize}
\end{frame}
%%%%%%%%%%%%%%%