% file: parts/partition.tex

%%%%%%%%%%%%%%%
\begin{frame}{}
  \begin{exampleblock}{Integer Partition ($\text{CS}: 1.5-4$ Extended)}
    What is the number of ways to pass out $k$ \red{identical} apples to \red{$n$-胞胎}. 
    Assume that a child may get more than one apple.
  \end{exampleblock}

  \pause
  \begin{columns}
    \column{0.50\textwidth}
      \fignocaption{width = 0.95\textwidth}{figs/apple-to-children-partition}
    \column{0.50\textwidth}
      \fignocaption{width = 0.95\textwidth}{figs/apple-to-children-partition-equiv}
  \end{columns}

  \pause
  \vspace{0.80cm}
  \centerline{\red{\large Integer partition of $k$ into $\le n$ parts}	\pause \quad \teal{(The order does not matter!)}}

  \pause
  \begin{theorem}[\uncover<6->{G. H. Hardy, Ramanujan (1918)}]
    \[
      p(k) \triangleq \sum_{x=1}^{x=k} p_{x}(k) \;\teal{\sim}\; \frac {1} {4\sqrt{3} k} \exp\left({\pi \sqrt {\frac{2k}{3}}}\right)
    \]
  \end{theorem}
\end{frame}
%%%%%%%%%%%%%%%

%%%%%%%%%%%%%%%
% \begin{frame}{}
%   \[
%     p_{n}(k): \text{ \# of partitions of $k$ into $n$ parts}
%   \]
% 
%   \begin{theorem}[Recurrence for $p_{n}(k)$]
%     \[
%       p_{n}(k) = p_{n-1}(k-1) + p_{n}(k-n)
%     \]
%   \end{theorem}
% 
%   \pause
%   \begin{proof}
%     \[
%       1 \le x_1 \le x_2 \le \cdots \le x_n
%     \]
% 
%     \begin{columns}
%       \column{0.40\textwidth}
% 	\[
% 	  \red{\text{\textsc{Case} } x_1 = 1}
% 	\]
% 	\[
% 	  1 = x_1 \le x_2 \le \cdots \le x_n
% 	\]
% 	\uncover<3->{
% 	  \[
% 	    \boxed{p_{n-1}(k-1)}
% 	  \]
% 	}
%       \column{0.60\textwidth}
%         \[
% 	  \red{\text{\textsc{Case} } x_1 > 1}
% 	\]
% 	\[
% 	  1 < x_1 \le x_2 \le \cdots \le x_n
% 	\]
% 	\uncover<4->{
% 	  \[
% 	    1 \le x_1-1 \le x_2-1 \le \cdots \le x_n-1
% 	  \]
% 	}
% 	\uncover<5->{
% 	  \[
% 	    \boxed{p_{n}(k-n)}
% 	  \]
% 	}
%     \end{columns}
%   \end{proof}
% \end{frame}
%%%%%%%%%%%%%%%
