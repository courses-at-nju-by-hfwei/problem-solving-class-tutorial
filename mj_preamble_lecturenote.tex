%for hyper link
%\usepackage[colorlinks, linkcolor=blue]{hyperref}

\usepackage[utf8]{inputenc}


\usepackage{latexsym}
\usepackage[T1]{fontenc}
\usepackage{lmodern}
\usepackage[english]{babel}
\usepackage{amsmath}
\usepackage{amsfonts}
\usepackage{amssymb}




%for graphs and pics
\usepackage{graphicx}
\usepackage{tikz}
\usetikzlibrary{matrix,shapes,arrows,positioning,fit,backgrounds,calc}
\usepackage{overpic}
%for animation
\usepackage[dvipdfmx]{animate}


%boxes
\usepackage[normalem]{ulem}
\usepackage[most]{tcolorbox}
\usepackage{fancybox}

\usepackage{bbding}
\usepackage{pifont}

%for tables
\usepackage{booktabs}
%for algorithms
\makeatletter
\newif\if@restonecol
\makeatother
\let\algorithm\relax
\let\endalgorithm\relax
\usepackage[linesnumbered,ruled,noend,noline]{algorithm2e}
\usepackage{algpseudocode}
\usepackage{amsmath}
\renewcommand{\algorithmicrequire}{\textbf{Input:}}  % Use Input in the format of Algorithm
\renewcommand{\algorithmicensure}{\textbf{Output:}} % Use Output in the 
\SetAlCapNameFnt{\scriptsize}
\SetAlCapFnt{\scriptsize}

% for Chinese
\usepackage{CJK}
\usepackage{zhnumber}



%for codes
\usepackage{listings}
\lstset{%
	alsolanguage=Java,
	%language={[ISO]C++}, %language为,还有{[Visual]C++}
	%alsolanguage=[ANSI]C, %可以添加很多个alsolanguage,如alsolanguage=matlab,alsolanguage=VHDL等
	%alsolanguage= tcl,
	alsolanguage= XML,
	tabsize=4, %
	frame=shadowbox, %把代码用带有阴影的框圈起来
	commentstyle=\color{red!50!green!50!blue!50},%浅灰色的注释
	rulesepcolor=\color{red!20!green!20!blue!20},%代码块边框为淡青色
	keywordstyle=\color{blue!90}\bfseries, %代码关键字的颜色为蓝色,粗体
	showstringspaces=false,%不显示代码字符串中间的空格标记
	stringstyle=\ttfamily, % 代码字符串的特殊格式
	keepspaces=true, %
	breakindent=22pt, %
	numbers=left,%左侧显示行号 往左靠,还可以为right,或none,即不加行号
	stepnumber=1,%若设置为2,则显示行号为1,3,5,即stepnumber为公差,默认stepnumber=1
	%numberstyle=\tiny, %行号字体用小号
	numberstyle={\color[RGB]{0,192,192}\tiny} ,%设置行号的大小,大小有tiny,scriptsize,footnotesize,small,normalsize,large等
	numbersep=8pt, %设置行号与代码的距离,默认是5pt
	basicstyle=\footnotesize, % 这句设置代码的大小
	showspaces=false, %
	flexiblecolumns=true, %
	breaklines=true, %对过长的代码自动换行
	breakautoindent=true,%
	breakindent=4em, %
	escapebegin=\begin{CJK*}{GBK}{hei},escapeend=\end{CJK*},
	aboveskip=1em, %代码块边框
	tabsize=2,
	showstringspaces=false, %不显示字符串中的空格
	backgroundcolor=\color[RGB]{245,245,244}, %代码背景色
	%backgroundcolor=\color[rgb]{0.91,0.91,0.91} %添加背景色
	escapeinside=``, %在``里显示中文
	%% added by http://bbs.ctex.org/viewthread.php?tid=53451
	fontadjust,
	captionpos=t,
	framextopmargin=2pt,framexbottommargin=2pt,abovecaptionskip=-3pt,belowcaptionskip=3pt,
	xleftmargin=4em,xrightmargin=4em, % 设定listing左右的空白
	texcl=true,
	% 设定中文冲突,断行,列模式,数学环境输入,listing数字的样式
	extendedchars=false,columns=flexible,mathescape=true
	% numbersep=-1em
}

\lstdefinestyle{styleJ}{
	language=[AspectJ]Java,
	keywordstyle=\color{keywordcolor}\bfseries, 
	commentstyle=\color{blue} \textit, 
	showstringspaces=false,
	numbers=left,
	numberstyle=\small
}
\lstdefinestyle{styleP}{
	language=Python,
	numbers=right, 
	frame=single,
	numberstyle=\small ,
}
\usetikzlibrary{decorations.pathreplacing}
\usetheme{Eastlansing}
\newcommand{\fallingfactorial}[1]{%
	^{\underline{#1}}%
}
\newcommand{\cmark}{\ding{51}}
\newcommand{\xmark}{\ding{55}}
\newcommand{\question}[2]{
	{\color{blue!60!black!90!}\large\textbf{Q #1}: #2}
}

%for colors

\usepackage{color,xcolor}
% predefined colors
\newcommand{\red}[1]{\textcolor{red}{#1}} % usage: \red{text}
\newcommand{\blue}[1]{\textcolor{blue}{#1}}
\newcommand{\teal}[1]{\textcolor{teal}{#1}}
\newcommand{\green}[1]{\textcolor{green}{#1}}
\definecolor{keywordcolor}{rgb}{0.1,0.1,0.8}
\definecolor{webgreen}{rgb}{0,.5,0}
\definecolor{bgcolor}{rgb}{0.92,0.92,0.92}

%for arrows
\usepackage{extarrows}

%for math function font
\usefonttheme[onlymath]{serif}